% LaTeX resume using res.cls
\documentclass[line,margin]{res} 

\usepackage{helvetica} % uses helvetica postscript font (download helvetica.sty)
%\usepackage{newcent}   % uses new century schoolbook postscript font 

\usepackage[brazilian]{babel}
\usepackage[utf8]{inputenc}
\usepackage[T1]{fontenc}
\usepackage{hyperref}

\begin{document}

\name{Bruno Croci de Oliveira}

\address{Av. Sete de Setembro, 1256 - Guarulhos - São Paulo}
\address{(11) 99932-6440 - bruno@croci.me - \href{http://bruno.croci.me/}{http://bruno.croci.me/}}

 
\begin{resume}
 
\section{OBJETIVOS} Trabalhar com programação de sistemas ou jogos, contribuindo,
                 assim, com o desenvolvimento da empresa bem como minha carreira
                 na área.
 
 
\section{EDUCAÇÃO} {\sl Bacharelado em Sistemas de Informação} \\
                Universidade de São Paulo, EACH-USP \\
                conclusão prevista: indefinido, curso trancado
                
                {\sl Técnico em Informática} \\
                Colégio Eniac \\
                conclusão: 2008 \\
 
 
\section{HABILIDADES} \begin{itemize}  \itemsep -2pt
                 \item Desenvolvimento Web com PHP (e Frameworks MVC), (X)HTML, CSS, JavaScript, 
                       Tableless e outros conceitos de WebStandards.
                 \item Banco de Dados SQL (MySQL e PostgreSQL) e não-relacionais (MongoDB).
                 \item Desenvolvimento de jogos nas plataformas Flash (ActionScript 3), Unity3D (C\#),
                       HTML5 Canvas (JavaScript) e C/C++ (com diversos Frameworks e APIs).
                 \item Implementação de servidores para jogos {\sl turn-based} com SmartFoxServer.
                 \item Linguagens Java e Python.
                 \item Ambientes baseados em Unix (Linux e Mac OS X) e ShellScript.
                 \end{itemize}
 
\section{EXPERIÊNCIA} {\sl Game Developer} \hfill 11/2011 - presente \\
                Hive Digital Media, São Paulo, SP
                 \begin{itemize}  \itemsep -2pt
                 \item Desenvolvimento de jogos em Flash (ActionScript 3 e ActionScript 2) para Web e outros dispositivos como Smart TVs.
		 \item Desenvolvimento de jogos em Unity3D (C\#) para mobile e web.
                 \item Programação PHP para servidor de jogos e aplicações para Facebook.
                 \item Java para servidores de jogos multiplayer com SmartFoxServer.
		 \item Desenvolvimento de aplicações mobile com Appcelerator Titanium (Javascript).
                 \item Estruturação e implementação de protocolos de comunicação para jogos multiplayer online.
                 \end{itemize}
                 
                 {\sl Lead Game Programmer} \hfill 09/2010 - 11/2011 \\
                Loopix Digital Group, São Paulo, SP
                 \begin{itemize}  \itemsep -2pt
                 \item Desenvolvimento de jogos em Flash com ActionScript 3.
                 \item Conceito e implementação de protocolos de comunicação para integrar jogos (em Flash e Unity3D) ao servidor (Java).
                 \item Programação de jogos multiplayer com o SmartFoxServer.
                 \end{itemize}
 
                {\sl Desenvolvedor Web} \hfill 01/2009 - 10/2009 \\
                Detetive.net, São Paulo, SP
                 \begin{itemize}  \itemsep -3pt
                 \item Desenvolvimento de sistemas para a Web com PHP e Zend Framework.
                 \item Implementação {\sl front-end} com XHTML, CSS e JavaScript (jQuery).
                 \end{itemize} 

\section{CAPACITAÇÃO}             
           {\sl Microsoft Students to Business} \\
                Centro de Inovação Microsoft, Senac, São Paulo \\
                conclusão: 2008 

\end{resume}
\end{document}
